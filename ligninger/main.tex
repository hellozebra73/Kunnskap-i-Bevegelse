\documentclass[12pt,twoside,onecolumn]{article}

\usepackage{a4}
\usepackage[margin=0.9in]{geometry}
\usepackage{pdfpages}
\usepackage{apacite}
\usepackage{jneurosci}
\usepackage[utf8]{inputenc}
\usepackage[T1]{fontenc}
\usepackage[norsk]{babel}
\usepackage{subfiles} % Handeling multiple chapters on seperate files.
\usepackage{graphicx}  % for displaying figures
\usepackage{subcaption}
\usepackage{url}
\usepackage{amsmath} % math package
\usepackage{relsize} % to use larger math symbols
\usepackage{amssymb} % for using blackboard letters (e.g R for real numbers)
\usepackage{tocbibind} % for adding the reference section in table of contents
\usepackage{fixltx2e} % for text in sub script
\usepackage{perpage} % for resetting footnote counter at each page
\usepackage{csquotes}
\usepackage{epigraph}
\usepackage{ragged2e}

\MakePerPage{footnote} %the perpage package command

\addto\captionsenglish{%
  \renewcommand{\figurename}{Figur}
}

\makeatletter
\def\@documentnocite#1{\@bsphack
  \@for\@citeb:=#1\do{%
    \edef\@citeb{\expandafter\@firstofone\@citeb}%
    \if@filesw\immediate\write\@auxout{\string\citation{\@citeb}}\fi
    \@ifundefined{b@\@citeb}{\G@refundefinedtrue
      \@latex@warning{Citation `\@citeb' undefined}}{}}%
  \@esphack}
\AtBeginDocument{\let\nocite\@documentnocite}
\makeatother

\begin{document}
\pagenumbering{roman} 
% Front page with uio logo - remember to fix the logo

\newpage\null\newpage
\pagenumbering{arabic} 


\section*{Ligninger}

\subsection*{Førstegradsligninger}

\textbf{Oppgave 1}
\begin{align}
7x - 3 &= 11  \qquad\text{ Plus 3 på begge sider av likhetstegnet.}\\
7x -3 +3 &= 11 + 3\\
7x &= 14\\  
\frac{7x}{7} &=  \frac{14}{7} \qquad\text{ Uttrykket kan forkenkles mer.}\\
x &= 2
\end{align}
\newline
\textbf{Oppgave 2}
\begin{align}
\frac{x}{2} + \frac{5}{6} &=  \frac{4}{3} - x \qquad\text{ Pluss x på begge sider av likhetstegnet.}\\
\frac{x}{2} + x + \frac{5}{6} &=  \frac{4}{3} - x + x\\
\frac{x}{2} + x + \frac{5}{6} - \frac{5}{6} &=  \frac{4}{3} - \frac{5}{6}\\
\frac{x}{2} + \frac{2x}{2} &= \frac{8}{6} - \frac{5}{6}\\
\frac{3x}{2}  &= \frac{3}{6} \qquad\text{ Uttrykket kan forkenkles mer.}\\ 
\frac{3x\cdot2}{7\cdot3} &=  \frac{1\cdot2}{2\cdot3} \qquad\text{ Uttrykket kan forkenkles mer.}\\
x &= \frac{1}{2}
\end{align}

\subsection*{Sett inn tall i formler}
\textbf{Oppgave 3}
Fatima har kjøpt nytt abonnement hos Telihor. I abonnementet har hun en fast beløp hver måned på 50 kr. I tillegg må hun betale 1.50 kr per MB hun bruker. 
\newline
\textbf{Del 1} Lag en funksjon som beskriver Fatima's total månedlig kostnad. La x værer antall MB hun bruker i måneden og P(x) hennes total kostnad per måned.
\begin{align}
P(0) &= 50 \text{ Hvis Fatime bruker ingen data, blir da hennes forbruk lik 50}\\ 
P(1) &= 50 + 1.5\cdot1\text{ Hvis Fatime bruker 1MB data, blir da hennes forbruk lik $50 + 1.5\cdot1$}\\ 
P(2) &= 50 + 1.5\cdot2\text{ Hvis Fatime bruker 2MB data, blir da hennes forbruk lik $50 + 1.5\cdot2$}\\ 
&\text{ Hva blir hennes forbruk hvis hun bruker x-antall data per måned} \nonumber\\
P(x) &= 50 + 1.5x
\end{align}
\newline
\textbf{Del 2}
Finn hennes total kostnad per måned når hun bruker 1000 MB = 1 GB data.
\begin{align}
P(1000) &= 50 + 1.5\cdot1000\\
P(1000) &= 50 + 1500\\
P(1000) &= 1550
\end{align}

\end{document}
